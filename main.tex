\documentclass[12pt]{article}
\usepackage{epsfig}
\usepackage[margin=1in]{geometry}
\usepackage{graphicx}
\usepackage{amssymb}
\usepackage{epstopdf}
\usepackage{fancyhdr}
\DeclareGraphicsRule{.tif}{png}{.png}{`convert #1 `dirname #1`/`basename #1 .tif`.png}
\usepackage{amsmath}
\usepackage{graphicx}
\usepackage{caption}
\usepackage{enumerate}
\usepackage{framed}
\usepackage[svgnames]{xcolor}
\usepackage{comment}
\usepackage[normalem]{ulem}
\usepackage{graphicx}
\usepackage{siunitx}
\usepackage{tcolorbox}
\usepackage[colorlinks=false,hidelinks]{hyperref} %%% Make this the last package to be loaded! %% included for a clickable toc.
\usepackage{subcaption}
\linespread{1.25}

\begin{document}

\title{Infrared Filtering for Large Aperture Millimeter Wave Telescopes}
\author{Albert Wandui \\
First Reader: Chao-Lin Kuo, Second Reader: Sarah Church}
\date{April 27, 2017}
% \twocolumn[
%     \begin{@twocolumnfalse}
\maketitle
\begin{abstract}
I worked on characterizing the performance of metal mesh reflective filters for large aperture millimeter wave telescopes. The biggest challenge is the large power through the window. Since we only have limited cooling power at the focal plane, we need to reject as much incident infrared radiation as possible. For BICEP3, we implemented metal mesh aluminum filters on mylar/BOPP substrates. This thesis is an investigation into the performance of these filters over the first three years of observation as well as lab testing and HFSS simulations to characterize their performance. We report that we can make filters with 0.99 transmission at 95 GHz and 0.97 transmission at 150GHz. With increase in frequency, the performance of the filters degrades at 220GHz. Fourier Transform Spectroscopy and X-Y scanning followed by cryogenic testing was used to check the performance of the filters. Recently began testing with foam filters which show more consistent characteristics across the entire filter surface which helps to mitigate systematic polarization effects in the data. We also assess the performance of double sided filters which are demonstrated to be an effect strategy to prevent radiation of the mylar substrates of the mesh filters. Finally, we present the future of large aperture IR filtering for CMB telescopes. 
\end{abstract}
%     \end{@twocolumnfalse}
% ]
\section{Introduction}

In order to make sensitive measurements of the millimeter wave sky, it is necessary for ground based telescopes to cool their detectors to lower the noise level to the photon shot noise limit. At the same time, efforts to image larger sections of the sky require placing more and more detectors on the focal plane and employing larger apertures for better sensitivity and depth in the measurements of the microwave sky. However, the increase in aperture size leads to an increase in the infrared loading of the telescope. Since these telescopes employ closed-cycle cryocoolers which have limited cooling power, this may easily overwhelm the cooling capabilities of the cryogenic systems. For a telescope such as BICEP3 this presents a serious challenge.

BICEP3 is a polarimeter that is focused on detecting the primordial gravitational wave B-mode signal from the Cosmic Microwave Background (CMB). A detection of this primordial B-mode signal would be direct evidence for inflation. The BICEP3 telescope is the latest addition in the BICEP/KECK series of CMB polarimeters based at the South Pole that observe the same patch of sky. So as to expand the existing dataset from BICEP2, which is at 150GHz, and Keck which is at 95GHz and 220GHz, BICEP3 makes its observations at 95 GHZ . Significantly, in comparison to BICEP2, BICEP3 has expanded the number of detectors on a single receiver by a factor of 5 to 2560 TES bolometers - enough to match the total number of detectors in all 5 receivers of the KECK telescope! In order to achieve such large detector counts on the focal plane, the aperture of the telescope was doubled to 520mm. However, this in turn led to an increase in the infrared loading through the window by a factor of 10 to about 150W of incident power. This is far too much for the first stage of the pulsetube cooling system, which is rated at 40W of cooling power at 45K. The use of large diameter metal mesh infrared filters is therefore one of the key technological developments that enables BICEP3 to make its observations.

Metal-mesh infrared filters are designed so that they are transparent at the microwave frequencies at which observations are made while cutting down significantly on incident infrared radiation (IR). In contrast to absorptive filters which absorb the infrared radiation, the metal mesh filters employed in BICEP3 are designed to reflect infrared radiation back out of the telescope receiver. In BICEP3, a stack of about 10 of these filters are mounted just behind the window and sit at room temperature. Through reflection coupled with some absorption, the filter stack effectively reduces the transmitted power to about 15W, well within the cooling capabilities of the 50K pulse tube. 

The large aperture of the filters needed has made previous fabrication techniques such as photolithography unscalable and increasingly expensive. Currently, photolithography is limited to filters with an aperture of about 30cm. As such, it was necessary to design and custom make filters. The IR filters consist of a thin plastic substrate with a uniform layer about 0.4 micron thick aluminum deposited. Both single and double sided filters are used in BICEP3, with double sided filters requiring aluminum layers on both surfaces of the dielectric. The choice of dielectrics was initially governed by the ease of procument but with further study, we chose dielectrics that were shown to have favorable in band characteristics. 3.5 $\mu$m thick Mylar, 6 $\mu$m Polypropylene/Polyethylene (PP/PE) and 6 $\mu$m thick Biaxially Oriented Polypropylene (BOPP) have been used as filter substrates. In order to make the filter, the aluminum on dielectric films were then subsequently mounted onto 26 inch diameter, 0.125 inch thick aluminum rings for structural support.  The frequency selective features of the mesh filters were made using a high power UV laser that ablated the deposited metal. This work was done at Laserod with the filters mounted on a precision XY stage during etching. For each filter, two orthogonal etching axes were defined and marked as X and Y on the ring. The X and Y markings were used to guide the laser ablation process. They defined the directions in which the microscopic mesh grid would be aligned. The initial development program of these filters was published in GIVE REF. These filters have been demonstrated to overcome the aperture limitations of photolithography and have been successfully deployed in BICEP3 for three observing seasons now. 


Despite their successful early implementation in BICEP3, additional research and development was necessary to overcome the challenges of deploying new technology. First, because the performance of the filters was tightly tied to the heat load balance of the cryostat, there was an increasing need to fully characterize the properties of the filters. In fact, analysis done on the cryogenic performance of BICEP3, during the first season, revealed that the filter stack sitting at ambient temperature, produced less reduction in infrared loading than would have been expected. Each filter that was deployed was designed to reflect about 50\% of the incident radiation at wavelengths of about 10 $\mu$m. As such, with N filters in a stack it was expected that the loading would decrease as $\left(\frac{1}{2}\right)^N$. Instead, the reduction in infrared loading was found to be similar to that expected from a series of radiative filters in thermal equilibrium. The reduction in performance could be attributed to non-negligible emissivity of the warm dielectric substrate. One possible solution to this problem would be to cool the filters to reduce their thermal emission which scales as the temperature T to the fourth power. Unfortunately, the dielectric substrates used to make the filters are poor conductors of heat. This coupled with the large aperture of the filter limits the amount of heat that can be carried away from the central regions of the filter if the filter is thermally anchored to a cold bath along its edges. A better solution to this problem is to deposit a second layer of aluminum to make a double sided mesh filter. The second layer reflects the emission of the dielectric layer right above it which has been shown to effectively lower the emissivity of the filter. A directed research and development effort was needed to fabricate, test and demonstrate the suitability of double sided filters in reducing thermal loading from the filters themselves. 

In addition, improperly fabricated mesh filters also had the potential to add to the non-CMB photon noise load of the detectors and introduce polarization dependent systematic effects. During the first observing season, several measurements showed a large mismatch in the optical transmission of the two orthogonal antenna polarization directions (denoted A and B) of the detectors. Newly fabricated mesh filters that were included in the filter stack were the key suspect. Measurements done on a polarization measurement setup revealed that some filters showed significant polarization dependent transmission. With the problem identified, the culprit filters were subsequently replaced and the expected performance was restored. However, this demonstrated the need for a systematic investigation into the quality, repeatability of production and improvement in the yield of filters. This thesis presents the results of such an investigation. We report on the currently established limits on the performance of laser ablated reflective metal mesh filters. In addition, we discuss the performance of the mesh filters over the last three observing seasons of BICEP3 with the aim of assessing the suitability of reflective metal-mesh IR filters for future millimeter wave telescopes.

% Ground based telescopes imaging the millimeter and sub-millimeter sky are increasingly employing cryogenically cooled bolometer arrays to reduce the detector noise level to the photon shot noise limit . With efforts to image more of the sky, these telescopes are employing larger and larger apertures to increase their sensitivity and view the microwave sky in unprecedented depth. However with a large aperture size, there is increased infrared heating of the focal plane which compromises the achievable sensitivity of the detector and overloading the cryogenic systems which have limited cooling power. This necessitates the use of infrared filters to reject unwanted out of band radiation while allowing in band wavelengths to reach the focal plane. However, with the increased aperture sizes, previous fabrication techniques used to develop these filters are no longer scalable. Metal mesh reflective IR filters have been successfully deployed in many ground based telescopes studying the millimeter wave sky. However, BICEP3 which is a third generation refractor telescope, has a large aperture which makes photolithography an expensive fabrication technique for IR filtering. Because of this, laser ablation of metal on dielectric substrate was developed as a cost effective alternative. Laser ablation was success-
% fully demonstrated to overcome the aperture limitations that were inherent in photolithography and laser ablated mesh filters have been successfully deployed in BICEP3 for three observing seasons now. However, the first generation of mesh filters showed polarization dependent transmission in our observing band. For the next observing season of BICEP3 and other future large aperture telescopes such as SPT-3G, a systematic investigation of the quality, repeatability in production and yield of the filters was needed. This paper is a report on the limits of the performance of laser ablated metal mesh filters as determined through in lab investigations as well as a discussion of their performance during the last three observing seasons of BICEP3. 

% In BICEP3, which is a 550mm aperture refracting telescope at 95 GHz, the reflective metal-mesh filters were used a stack just after the HDPE vacuum window. The metal mesh sliders sat at 300 K and were used to cut down on most of the incident radiation from room temperature incident through the window. BICEP3 receives about 165W of IR loading through the window. We only have about 40W of cooling at the 50K stage and about 1W at the 4K stage. The metal mesh filters cut down about 65W of IR loading. The mesh filter stack is followed by an alumina filter at 50K to cut down on the emission from the warm mesh filter stack. The alumina lenses sit at the 4K stage as well as a nylon filter. The combination of metal mesh filters and the cold alumina and nylon filters cut down the incident power to about 5 $\mu$W at the sub-kelvin stages.

% The filters are made of about 0.04 $\mu$m of aluminum metal deposited on dielectric substrates. For BICEP3, we used 3.5 $\mu$m thick Mylar, 6 $\mu$m BOPP and 6 $\mu$m PP/PE as the dielectric substrates. BOPP and PP/PE were tested because Mylar was found to have strong absorption peaks in the far infrared which would make it emissive at the frequencies we are trying to get rid of. Using a frequency-tripled Nd:YAG laser with 355 nm wavelength, directed by a mirror galvanometer, avenues are etched into the metal layer. The laser power, write speed and spot size can be accurately tuned to achieve the desired mesh filter characteristics. For BICEP3, only capacitive mesh filters with islands of metal squares were used.


\section{Theory of Metal Mesh Filters}
   
The earliest work done on the optical properties of metal mesh filters was done by Ulrich. An ideal capacitive metal mesh filter consists of infinitely thin grid made of perfectly conducting metal. In Ulrich's initial analysis the effect of the dielectric substrate was neglected. In this idealization, the characteristics of the mesh are perfectly described by two numbers; the grid period, $g$ and the gap width between the metal squares, 2 $w$. Consider a plane wave incident normally on the grid. Intuitively, for wavelengths, $\lambda > g$, we can ignore diffraction effects and the mesh is transparent to the incident wave and the wave is fully transmitted. For $\lambda < g$, surface currents are setup in the metal squares. These surface currents generate an electromagnetic field that interferes destructively with the incident plane wave and reducing the transmission to zero. The incoming wave is thus fully reflected. We can also expect that for $\lambda \approx g$, the packing ratio of the squares to the gaps between the squares determines the proportion of the incident field that is transmitted through. Defining a normalized frequency $\omega = g/\lambda$, we can summarize the properties of the mesh filters as 

\begin{enumerate}
    \item for $\omega > 1$, the wavelength is small and the incident wave sees each metal square as an infinite plane. The incident wave is purely reflected as a mirror would.
    \item for $\omega < 1$, the wavelength is large and the incident wave does not see the metal mesh squares and is purely transmitted. In this regime we can ignore diffraction effects.
\end{enumerate}  

From this, we see that the mesh filter acts as a low-pass filter. This is the reason why such filters are referred to as capacitive mesh filters. In the $\omega < 1$ regime, we can thus model the effect of the mesh filter by constructing an electrical line equivalent circuit. In this representation, transmission lines represent free space while the mesh filter is a lumped circuit element that shunts the transmission line. As a first approximation we can describe the mesh filter using a single number, the capacitance of the equivalent circuit, C. The equivalent circuit representation is shown in the figure below. 

This simple model is inadequate in describing the frequency response of the filter as $\omega \rightarrow 1$. From actual measurements of capacitive mesh filters, it is known that the transmission of the capacitive filter approaches zero in this limit. This implies that capacitive filters have resonance feature at or near $\omega = 1$. These resonant features are not adequately captured by the simple capacitive admittance model. The model is improved by adding an inductive element, L to describe the resonant frequency. Adding a resistive element, R, in series with the capacitance also describes the effect of ohmic losses in the metal squares of the filter. The mesh filter is thus described by its resonant frequency, $\omega_0$, characteristic impedance, $Z_0 = \omega_0 L = 1/\omega_0 C$ and its resistance R. The improved circuit representation is shown in the figure below.

Since the spectral characteristics of the mesh filter are functions of the grid period, g and the gap width w (as long as we can neglect the thickness of the squares,  we can define a dimensionless shape parameter, $w/g$, which captures the features of a particular capacitive grid. It then follows that the resonant frequency and the characteristic impedance are functions of the shape parameter, $\omega_0 = \omega_0 (w/g)$ and $Z_0 = Z_0 (w/g)$. The full functional dependence can be obtained by fitting for the resonant frequency and characteristic impedance of many filters with different shape parameters. Instead of obtaining the full functional dependence, it is usually easier to make measurements of a few filters with different shape parameters to figure out more general dependence. From such measurements, we concluded that the position of the resonant peak is dependent on the value of g, while the shape parameter controls the steepness of the cutoff. With larger values of the shape parameter, the cutoff is steeper.

By tuning the values of $g$ and $w$, we can obtain filters with excellent infrared rejection capabilities. The bulk of the infrared radiation incident on the telescope is from the sky at 300K with a peak at around 10 $\mu$m. A capacitive filter with a grid period on the order of 10 $\mu$m is well suited to rejecting the 300K blackbody peak. For BICEP3, mesh filters of three different shape parameters were tested and modelled. The shape of the mesh filter is described as a/b where a is the size of the mesh square and b is the gap between two adjacent squares. $a + b = g$ and $b = 2 w$. The figure below shows the modelled reflectance of the mesh filters across a wide frequency range. For comparison, the 300K blackbody spectrum has also been plotted to highlight the expected rejection of the blackbody peak for each of the three filters. As expected from the equivalent circuit model, the resonant peaks of the filters are found at wavelengths on the order of the grid period.

In the $\omega > 1$ regime, we can no longer ignore diffraction effects. In addition, the effects of non-normal incidence as well as losses in the dielectric become important. Then the spectral response of the filters can be obtained using more detailed modelling. One of the modeling tools used for the BICEP3 filters was High Frequency Structure Simulator (HFSS). HFSS solves Maxwell's equations for a unit cell of the mesh structure allowing accurate prediction of the spectral response of the filter. Other modeling techniques such as Floquet analysis which is an analytical method for computing the transmission through the diffraction regime are also used in filter work. However, Floquet analysis was not used in modeling the mesh filters described here.

Even without detailed modeling, we can still obtain some intuition for the behavior of the mesh filters for the high frequency limit where $\omega \gg 1$. Since $\lambda \ll g$, the incident photons see either the transmissive paths made by the dielectric gaps between the squares or instead the perfectly reflective metal squares which act as mirrors. We then expect that the reflectance, R of the filters in this limit is governed by the fraction of the total area of the filter that is covered by the metal squares. The reflectance of the mesh filter is therefore given by the relation $R \approx \left(\frac{a}{g}\right)^2$ which gives R $\approx$ 40\% for the shape parameters of the BICEP3 filters.

% In the $\omega < 1$ regime, we see that the metal mesh filters acts as a low as a first approximation, we can model the frequency response of the mesh filter as a capacitive low pass filter with a capacitance C. The value of the capacitance is adopted to fit the measured transmission of the mesh filter. Using the lumped circuit model, we can describe the effect of the mesh filter by an electrical line equivalent circuit. The first approximation is
% From this,we see that the transmission characteristics of the metal mesh filters can thus be modelled as a capactive low-pass filter. The impedance of the mesh filter can be tuned by changing the grid period and the gap width. While this analysis provides a good basis for understanding the general charcteristics of the metal mesh filter, the simplicity of the lumped element model is not sufficient to describe the effect of the losses due to the dielectric substrate, non-normal incidence of the radiation or behavior of the filters in the diffraction region. To fully account for these effects, other modelling tools are necessary. For BICEP3, High Frequency Structure Simulator (HFSS), which provides numerical solutions to the full Maxwell's equations was used to determine the expected filter characteristics for mesh filters of different impedances.       

% The HFSS simulations show that the mesh filters have a resonant reflective peak at about 10 $\mu$m, which is the peak wavelength of 300 K blackbody radiation. For BICEP3, the filters are labelled as a/b where a is the size of the grid square and b is the gap between adjacent squares. From the simulations, we expect that in our bands of interest; 95, 150 and 220 GHz, the total reflection would be on the 1\% level. At about 10 $\mu$m, we expect $R \approx \frac{a^2}{(a + b)^2}$, which is the fraction of the total area of the filter that is reflective. Indeed, this is what we see from the simulations.

The dielectric substrate can significantly affect the properties of the mesh filter. If the dielectric is highly absorptive at the peak of the 300K blackbody, then it will absorb incident radiation and re-emit it. The lower levels of the optical train will thus see a very warm filter. As mentioned before, Mylar was found to have absorption peaks in the far infrared. To compensate for the added loading due to the emissivity of Mylar, we used thin 3.5 micron Mylar sheets as the substrate for the filters. In contrast, BOPP and PolyPropylene/PolyEthylene (PP/PE) filters could be much thicker at about 6 microns. The thicker substrate is advantageous because it allows for the deposition of Aluminium on both sides of the substrate to produce double-sided filters. 

For effective filtering multiple filters must be used in a stack. The filters are randomly oriented with respect to each other to prevent an alignment of features of the mesh that could lead to higher resonant transmissions of out of band frequencies. In BICEP3, about 8 filters mounted on rings were used in the telescope for each season. Each filter was visually inspected and the total transmission of the filter stack was measured before the stack was used.  

% \section{Initial Challenges of Producing Mesh Filters}
% Initial testing of the earliest generations of mesh filters were done in lab on a turn table setup. This work was carried out by Keith Thompson and Kimmy Wu later at the pole. The turntable setup consisted of the source and detector aligned vertically with the sample placed in between the source and detector. The etching axes of the mesh filter could be aligned with the polarization of the source for co-polar response or clocked at an angle away. A $\pi/2$ clocking angle would give the cross-polar response. Initially, the sample was tested on axes but since the mesh filters were flat, off-axis testing was also possible to measure the transmission of various parts of the filter surface. The turntable was driven by a stepper motor which rotated the sample at a fixed rate through the field of the radio frequency source.

% These initial tests revealed that the earliest filters showed a 2-6\% variation in the transmission across the surface of the filter. The tests also showed that the transmission of the filters was polarization dependent. This was however demonstrated to be a very local effect. This early investigations led to a need to fully characterize the filters and assess their quality in order to select the best filters for the instrument. 

% In order to perform further testing, we transferred sections of the larger 25" filters onto smaller 6" mounting rings. The 6" rings were designed to be versatile and capable of being used in many different optical testing setups. 
% % \section{Fabrication Process for Mesh Filters}
% % The fabrication process for mesh filters was


\section{Optical Characterization of Mesh Filters}
Before the aluminum on dielectric film was mounted on the tensioning ring during fabrication, it was visually inspected for surface irregularities. With especially thin dielectric films, we noted that there was high probability that the film would have already been strained beyond yield during manufacture. This was evidenced by the presence of persistent wrinkling of the film that were aligned in linear features. Such films are difficult to uniformly etch and thus could only be used if a large enough section that was wrinkle free was found. In addition, we performed surface resistance measurements to ascertain that the deposited layer of aluminum was uniform and of the right thickness. 

After fabrication, we used an optical microscope setup to check for imperfections in the etched grid. The setup consists of a camera mounted on top a tube with an objective lens at the bottom. The camera and lens tube were mounted on a sliding track to allow the microscope to scan over the surface along one axis without moving the filter. The images were magnified by a factor of 100 to make the mesh elements visible and distinct. We also ascertained that the dimensions of the grid were as per the design specifications. As the microscope scanned over the filter, multiple images were taken. However the resolution at high magnification was limited since the setup was very sensitive to the small persistent vibrations of the table. These vibrations were largely due to the air currents in the clean room.

We did not inspect the entire filter surface but instead chose several spots to examine closely. For each spot chosen, we explored a surrounding region of about 5 cm radius by sliding the microscope on the track. We noted a number of different imperfections that could form during the fabrication process. The two main defects are illustrated in figure X. The most common defect found, was the bridging of adjacent metal squares by thin strips of metal along the edges of the squares. Given the two axes of the mesh, X and Y, we noted that there was preferential bridging along the Y axis of the filter. This feature was responsible for the polarization dependent transmission of the filters. If the polarization vector is aligned with the axis of the bridge, then it can setup surface currents that cross from square to square reflecting the incident wave and lowering the transmission. Once these defect was identified, we worked with Laserod to increase the stability and focused power of the laser to ensure that the mesh avenues were fully ablated. While these features were common in earlier generation filters, they were largely absent in the filters made for the third observing season of BICEP3. The preference for bridging along Y as opposed to X was likely due to fact that the Y axis was always etched first.

A less common but more severe defect would be the complete unetching of the avenues between adjacent squares. Such defective regions had no transmission even in the science band since incident waves would see a purely metal surface. However, the low transmission of these filters in band made them easy to identify by measuring the transmission of the entire filter at high resolution and noting regions of unusually low transmission. The ease of identification meant that such filters were always eliminated early and not used in any science measurements. The high resolution transmission measurements are described further in section YYY.


\section{Spectral Measurements}
Experimental measurements of the mesh filters were made at 95 and 150 GHz using a polarizing Fourier Transform Spectrometer. 

The goal is to characterize our mesh filters in the 95, 150, 220 GHz bands. In order to do this, FTS measurements of mesh filter etalons were done at 95 and 150 GHz. The mesh filters are being sent to Harvard for measurements at 220GHz. From the FTS measurements, we extracted the attenuation due to loss + scattering (A), reflectance (R) and transmission (T) of the filters. Keith has a great discussion of the working of the FTS setup we have in the lab in his logbook entry on Alumina Tests. There is also a discussion on FTS measurements of mesh filters that was written as part of the mesh filter testing for BICEP3 last season here. For my analysis, I found the reference book: Introduction to Fourier Transform Spectroscopy by Robert John Bell to be particularly useful in the analysis.
For this analysis, two mesh filters were selected: The first was 18-01 which is a 15/40 single sided filter brought back from the Pole by Jae Hwan. From turntable tests done at Pole, 18-01 showed a 1-2\% polarization in the 2 θ co-polar response. The second filter selected for FTS testing wa s15-02 which is a double sided BOPP 15/40 filter. In addition, I will be testing 23-01 which is the 15/40 filter that showed a 5\% polarization effect in the 2 θ co-polar response from turntable tests at the south pole. I will be discussing the comparison between the turntable and the FTS results later in this posting.

For easier testing at the FTS, Ki Won designed and obtained some 6" rings onto which we transferred the mesh filters. Each 6" ring has 2 x 2 countersink holes, 2 x 2-56 tapped holes, and 4 x 4 clear holes. The transfer was done so as to ensure that the polarization direction of the mesh filters was preserved accurately onto the smaller mesh filters. In addition, I ensured that the filters were not additionally strained during the process. The thickness of a 1" piece of zote foam was sufficient to ensure that there was no additional straining on the filter during the transfer. A thin veneer of rubber cement was used to attach the filters onto the new rings once the transfer was done. The X and Y polarization directions were marked out onto the zote foam piece that held the mesh filters. Since both 6" filters were oriented in the exact same way, by aligning the zote foam piece with the polarizations on the mesh filter, the polarization directions were preserved.
For the 18-01 single sided mesh filter, an etalon was made with a 2" spacing between the two filters. This translates to a length of the etalon cavity of 2.25" between the actual filter surfaces accounting for the thickness of the rings. For the 15-02 double sided BOPP, the spacing was reduced to 1.5". I also designed a holder for the mesh filters to ensure that the filter surface was perpendicular to the beam direction and that both X and Y polarization measurements could be easily made. When the etalon was constructed, care was taken to ensure that the mesh filter rings were aligned with each other. Here are some pictures of the 6" rings with the filters transferred plus one of the etalons made using the filters.The FTS was used in Mode A which is discussed in depth in Keith's: Introduction to FTS testing. In mode A, the etalon is placed before the beam splitter. Alternate scans of sample vs no-sample were taken for each polarization direction.

I took a total of 7-9 scans; alternating between scans with a sample and scans without a sample and ending on a no-sample scan. When the FTS is used in Mode A, the interferogram that is obtained is simply the autocorrelation function of the electric field from the source. In this case, by taking the fourier transform of the interferogram we obtain the Power Spectral Density (PSD) of the source which is essentially the spectrum we are looking for. The autocorrelation function ought to be perfectly symmetric. Our interferograms depart from this in two significant ways. First, it seems that the power output of the FTS source drifts with time. Keith mentioned that this may be due to the source heating up with time. Because of this, the interferogram generally has a downwards slope from left to the right. In addition to this, during many of the initial scans, there was an additional feature on the right one side of the spectrum. The interferogram below shows an example of this. This particular interferogram was from the initial 15-02 605,606 etalon scans at 95 GHz. In order to correct this, before starting the actual data collection, I make a number of one-arm scans. During the one-arm scans, the fixed mirror is completely blocked with a piece of eccosorb. Therefore, the one arm scan simply measures the instrumental response of the moving arm. By subtracting off the one-arm scan from the two-sided interferograms, I can eliminate the drift and additional features in the interferogram. In the earliest scans that I did, I did not take any one-arm scans. I achieved the same drift correction by computing a moving average of the data with the averaging window large enough for the oscillations in the interferogram to cancel out. This extracts out the drift of the source which can then be subtracted away. Below is also a plot of the same interferogram as before with drift correction.

Once the interferograms are drift corrected, they next step involves aligning the interferograms from multiple scans. Why is this important? This is because the point of zero path difference, does not coincide for all scans that are taken. However, in order to correctly take the fourier transform, the exact point of zero path difference (pd) needs to be included in the data. Failure to do so leads to all sorts of phase errors that complicate the analysis. It's easy to identify the peak by fitting a quadratic to the three points centered around the max value that the interferogram takes. Once the peak is obtained, the encoder values can be converted to represent the path difference. The plot below shows the variation in the location of the point of zero pd over multiple scans. Fitting for the peaks allows one to align the interferograms with each other before taking the fourier transform.

In order to correctly compute the fourier transform and take its real part as the spectrum, the data array needs to be circularly symmetric: if the interferogram is represented by the array $x[i]$ where i indexes the array, then the condition $x[i] == x[-i]$ must hold. It then follows that if the interferogram is an array of length $2^n$, then 
$x[2^{n-1}]$ is the point of zero path difference if the array indexing starts from zero. Using this conditions, I resampled the interferogram to be an array of $2^n$ long with n=15. I also extended the length of the interferogram by zero padding it beyond the last data points. This extra length increases the resolution of the fourier transform allowing for better defined spectra. The plot below shows the extended resampled interferogram in comparison to the actual interferogram obtained from the fts scans.

With all these corrective steps in place, we can now take the real part of the fourier transform as the spectrum resulting from the interferogram. The imaginary part of the fft is not exactly zero due to noise features in the data. Keith had mentioned that you could use the imaginary part of the fft to assign error bars to the data but I did not include this step. Instead, I combined the spectra obtained over multiple scans and used their standard deviation as an estimate for the size of the errorbars. This is shown in the two plots below. The first is the spectra extracted from the spectra when there was no sample in the beam. There is very little deviation from scan to scan. The second plot details the spectra with the 15-02 605, 606 etalon in the beams. There is a much greater deviation in the spectra leading to larger errorbars. Again, note that the errors do increase towards the edges of the band. This was an important consideration for the final fitting of the fringes.

The last step before fitting is taking the ratio of the combined spectra with the sample in the beam to the combined spectra with no sample in the beam. Figure below shows the fringes obtained. I've also included the error bars to show the variation in the ratio. There are a couple of things that I do not entirely understand about the fringes. First, the amplitude of the oscillations varies dramatically. At the edges of the band the peak to peak amplitude of the fringes is large about 0.1. In the middle of the band, the fringes become irregular with an amplitude about 0.05. This made it particularly difficult to try and fit the fringes to a particular form because the behavior of the fringes changes across the band. In addition, the phase of the fringes shifts slightly as you sweep across the bands. A fit that matches the initial peaks slowly drifts out of phase with the peaks further down in the band. My suspicion is that there may be some noise artifacts that have filtered into the data that I took.

In this last section, I will describe the fiting procedure that I implemented for this analysis. The Fabry Perot fringes are described by the equation $f(\nu) = A \left[1 + F \sin^2\left(B \nu + \phi\right)\right]^{-1}$. Here $A = \left(\frac{T}{1-R}\right)^2$, while $F = \frac{4 R}{(1- R)^2}$ and $B = \frac{4 \pi}{c} L$. In the equation above, $\nu$ is the frequency, R is the fraction reflected, T is the transmission fraction and L is the effective length of the etalon (I've folded in any angles into the effective length). $\phi$ is a phase factor due to phase changes on reflection. I am not entirely sure of what the phase factor should be for reflections off the mesh filters so I left it as a free parameter. Of interest would be to check that the phase obtained after fitting matches the expected phase change.
We can use Maximum Likelihood Estimation (MLE) to find the best fit parameters. Given the ratio, $y$, frequency, $\nu$ and fit model, $f(\nu, \vec{\theta})$, we can compute the log likelihood function $\mathcal{L}\left(y| \nu,\sigma, \vec{\theta}\right) = \Sigma_{i} \left(\frac{y_i - f(x_i, \vec{\theta})}{\sigma^2_i}\right)$. $\vec{\theta} = [R, T, L, \phi]$ is the vector of the fit parameters and $\sigma_i$ is the error at each data point i. Here, I assumed that the errors were gaussian hence the form of the fit function. To find the best fit parameters, I maximized the log likelihood function (I actually minimized the negative of the log likelihood function). For a good fit, the log likelihood function is simply the chi square, so it should have a value of about 1.0 at best fit. This was however not the case for most of the fits that I did.

Assuming the fits were great, what are the uncertainties on the values of the parameters extracted? To answer this question, I used Maximum a Posteriori Estimation (MAP). The probability of seeing the parameters $\vec{\theta}$ given our data, errors and model is given by $p (\vec{\theta}| x,y, \sigma) = p (\vec{\theta}) p (y| x, \sigma, f, \vec{\theta})$. In the MLE estimation above, we already wrote down the likelihood function for $p (y| x, \sigma, f, \vec{\theta})$. The missing piece of the puzzle is the prior probability of the parameters. Since, I really had no prior knowledge on the distribution of the parameters, I used uniform (ie: uninformative) priors. R and T were limited to the range [0,1] while the effective length was given a range from [0, 2*actual length of etalon]. With this setup, I now have a distribution that I can sample over using Markov Chain Monte Carlo (MCMC). The actual implementation I used was a nice packaged python program from MIT called emcee. I initialized about 200 random walkers in a small ball around the results of the MLE estimation. Then I ran the Markov chain for about 1000 steps and plotted all the one and two dimensional projections of the parameters. These results are illustrated in the plot below. The two dimensional projections illustrate the covariances between the parameters and the one dimensional projections are the marginal distributions of the parameters. Finally using the marginal distributions, I used the 16 and 84th percentiles as the uncertainties in the final values of the parameters.

\section{Measurements of Local Polarization Dependent Variations across Filter Surfaces}
Since the FTS measurements give the averaged transmission of a pair of mesh filters, it was necessary to compare the results from the FTS measurements to direct measurements of the transmission of the filters.

\section{Summary of the Results and a Comparison with the Theoretical Predictions}

\section{Foam Filters and Prospects for the future of IR Filtering for Millimeter Wave Telescopes}

\section{Conclusions}

\end{document}